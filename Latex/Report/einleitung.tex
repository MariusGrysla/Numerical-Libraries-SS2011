\section{Einleitung}

Im Rahmen dieser Arbeit werden wir uns mit den Funktionen beschäftigen, welche die {\it NAG C Library}, eine Sammlung numerischer Algorithmen der {\it Numerical Algorithms Group} für die Programmiersprache {\it C}/{\it C++}, ihren Anwendern zur {\it Korrelation} und zur {\it Regressionsanalyse} zur Verfügung stellt.\\
Die Begriffe "`Korrelation"' und "`Regressionsanalyse"' sind der {\it deskriptiven Statistik} zuzuordnen, welche sich mit der Beschreibung und Darstellung von Daten befasst, die zuvor durch Erhebungen (z.B. Befragungen) oder Experimente gewonnen wurden.\\
Ein Beispiel für eine Erhebung wäre etwa die Studentische Lehrveranstaltungsbewertung, die an der {\it RWTH Aachen} regelmäßig durchgeführt wird. Die bei Erhebungen und Experimenten gewonnenen Daten umfassen {\it Merkmale} bestimmter {\it Merkmalsträger}: In unserem Beispiel wären die Merkmalsträger unter anderem die Studenten, die an der jeweiligen Erhebung teilnehmen, da sie zu mehreren persönlichen Merkmale befragt werden -- nämlich zu ihrem Geschlecht, ihrem Fachsemester und ihrer Nationalität. Weitere Merkmalsträger sind selbstverständlich die Lehrveranstaltung, der Dozent sowie die Rahmenbedingungen, deren unterschiedliche Merkmale von den Studenten bewertet werden. Die einzelnen Merkmale können verschiedene {\it Merkmalsausprägungen} annehmen: Das Merkmal "`Fachsemester"' eines Studenten kann bspw. die Ausprägungen "`1-2"', "`3-4"', ..., "`9-10"' oder "`über 10"' annehmen. Die Merkmale der Lehrveranstaltung und des Dozenten können von den Studenten überwiegend auf einer fünfstufigen Skala von "`trifft völlig zu"' bis "`trifft nicht zu"' oder mit Noten von "`sehr gut"' bis "`mangelhaft"' bewertet werden, wodurch sie eben diese möglichen Ausprägungen erhalten.\\
Im Anschluss an eine Erhebung oder ein Experiment lässt sich  der jeweils gewonnene Datensatz wie bereits erwähnt mit Hilfe der deskriptiven Statistik auf unterschiedliche Art und Weise beschreiben und darstellen. Eine Möglichkeit ist es, ein einzelnes Merkmal zu betrachten und z.B. den Mittelwert der Noten zu berechnen, welche die Studenten der Lehrveranstaltung gegeben haben. Dies ist das Feld der {\it univariaten Statistik}. Eine weitere Möglichkeit ist es, zwei oder mehrere Merkmale gleichzeitig zu betrachten, um sie auf eventuelle Zusammenhänge zu prüfen. Es wäre bspw. denkbar, dass das Merkmal "`Fachsemester"' eines Studenten und seine Bewertung des Merkmals "`Schwierigkeitsgrad"' der Lehrveranstaltung auf gewisse Weise zusammenhängen, was wichtig für die Interpretation des letzteren Merkmals wäre. Dies ist das Feld der {\it bi-} bzw. {\it multivariaten Statistik}. Hier sind schließlich auch die Korrelation und die Regressionsanalyse einzuordnen, welche sich beide mit der Beschreibung zweier oder mehrerer Merkmale und ihres Zusammenhangs beschäftigen.

\subsection{Motivation}

Es ist natürlich nicht nur im Rahmen der Studentischen Lehrveranstaltungsbewertung an der RWTH Aachen interessant, die möglichen Zusammenhänge von Merkmalen festzustellen. Die Anwendungsbereiche der Korrelation und der Regressionanalyse erstrecken sich von der Wirtschaft (Wie wirken sich Werbemaßnahmen auf die Verkaufszahlen eines Produkts aus?) über die Medizin (Durch welche genetischen Veranlagungen wird eine Krankheit begünstigt?) bis hin zu Sozial- und Naturwissenschaften, um nur einige Beispiele zu nennen. Auf Grund der vielfachen Anwendung ist es wünschenswert und oft sogar nötig (etwa bei sehr großen Datensätzen), solche statistischen Berechnungen automatisiert ausführen zu lassen. Die {\it NAG C Library} ist dabei ein wichtiges Hilfsmittel, stellt sie doch viele der nötigen Funktionen bereits in fertiger Form zur Verfügung, so dass sie in beliebige Anwendungsprogramme integriert werden können -- vorausgesetzt, sie wurden in {\it C}/{\it C++} oder einer anderen unterstützten Programmiersprache geschrieben.