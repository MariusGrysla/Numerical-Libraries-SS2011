\section{Zusammenfassung}

In dieser Arbeit haben wir eine Einführung in die Funktionalität und die Benutzung mehrerer Funktionen der \naglib gegeben.
% TODO: Noch genauer werden?

Insgesamt hat die \naglib einen deutlich größeren Funktionsumfang im Bereich Korrelation und Regression als die GSL.
Zum einen werden mehr verschiedene Berechnungsmethoden unterstützt und zum anderen existieren Funktionen die bestimmte Aufgaben erleichtern, wie beispielsweise die Berechnung aller Residual-Quadratsummen zur Bestimmung des optimalen Regressionsansatzes.
Zudem erwiesen sich die behandelten Funktionen der \naglib in vielen Fällen als performanter als ihre Gegenparts.
Ein weitere wichtiger Aspekt für viele Benutzer von numerischen Bibliotheken sind die Tests die eine Funktion durchlaufen hat.
In besonders kritischen Anwendungen, wie zum Beispiel Anwendungen für den Finanzmarkt, ist es darüberhinaus wichtig einen gewissen Testumfang garantiert zu bekommen.
Dies kann die \naglib bieten, während die GSL hier keinerlei Garantien geben kann. 
Zusammenfassend kann man daher sagen, dass die \naglib sehr gut für Anwendungen im Bereich der Korrelation und Regression geignet ist.

%\subsection{Ausblick}
% TODO: Benötigen wir einen Ausblick? Falls ja: Was reinschreiben? 

%%% Local Variables: 
%%% mode: latex
%%% TeX-master: "report"
%%% End: 
