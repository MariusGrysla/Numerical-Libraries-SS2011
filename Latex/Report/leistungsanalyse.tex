\section{Leistungsanalyse}
\begin{comment}
Die zwei wichtigsten Qualitätskriterien für mathematischen Funktionen sind Korrektheit und ein möglichst geringer Verbrauch von Ressourcen.
Eine gute Funktion sollte daher für alle möglichen Parameter den mathematisch korrekten Wert berechnen und dies zudem mit möglichst wenig Ressourcen, wie Speicher oder Prozessor-Laufzeit.
Da die {\it NAG C Library} bereits seit 1990 entwickelt wird\cite{wikipedia:nag} und zudem alle Methoden vor der Veröffentlichung auf Korrektheit geprüft werden\cite{NAG2011}, ist nicht davon auszugehen, dass Fehler vorhanden sind.

Dazu sollen zwei Algorithmen der Bibliothek auf ihren Ressourcenverbrauch untersucht werden: Der Bravais-Pearson Korrelationskoeffizient und die multiple lineare Regression.
Für beide werden Laufzeit und Speicherverbrauch unter Benutzung von verschiedenen Parametern mit ansteigender Komplexität untersucht.
% Damit alle Berechnungen vergleichbar sind, werden sie auf dem RWTH Cluster ausgeführt.
\end{comment}