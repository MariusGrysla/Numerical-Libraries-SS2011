\section{Korrelation}

Der Begriff {\it Korrelation} bezeichnet einen Zusammenhang zwischen zwei oder mehreren Merkmalen. Man sagt auch, dass diese Merkmale {\it korrelieren}. Wie stark dieser Zusammenhang ist, lässt sich messen, indem man verschiedene sog. {\it Korrelationskoeffizienten} berechnet. Im Folgenden werden wir zwei dieser Korrelationskoeffizienten vorstellen und im weiteren Verlauf der Arbeit näher untersuchen, wie ihre Berechnung in den entsprechenden Funktionen der {\it NAG C Library} implementiert ist.

\subsection{Korrelationskoeffizient nach Bravais und Pearson}

Der Korrelationskoeffizient $r$ nach Bravais und Pearson (auch {\it empirischer Korrelationskoeffizient} oder {\it Produkt-Moment-Korrelation} genannt) misst die Stärke des \underline{linearen} Zusammenhangs zweier Merkmale. Dabei ist zusätzlich anzumerken, dass beide Merkmale mindestens in {\it intervallskalierter} Form vorliegen müssen, d.h. alle möglichen Merkmalsausprägungen müssen als Zahlen interpretierbar sein, deren Abstände genau messbar sind. Der Wertebereich von $r$ ist $-1 \leq r \leq +1$. Der für zwei Merkmale berechnete Wert von $r$ ist so zu interpretieren, dass $r > 0$ positive, $r < 0$ negative und $r = 0$ keine lineare Korrelation bedeutet. In der {\it NAG C Library} dient die Funktion nag\_corr\_cov zur Berechnung dieses Korrelationskoeffizienten.\\
Liegen die Merkmale, deren Korrelationsstärke gemessen werden soll, nur in {\it ordinalskalierter} Form vor, d.h. sind alle möglichen Merkmalsausprägungen lediglich nach einer Reihenfolge sortierte Kategorien, kann auf den folgenden Korrelationskoeffizienten zurückgegriffen werden.

\subsection{Rangkorrelationskoeffizient nach Spearman}

Der Rangkorrelationskoeffizient $r_{SP}$ nach Spearman misst die Stärke des \underline{mono-}\\\underline{tonen} Zusammenhangs zweier mindestens ordinalskalierter Merkmale, indem die ursprünglichen Merkmalsausprägungen vor der eigentlichen Berechnung in {\it Ränge} transformiert werden. Der eigentliche Rangkorrelationseffizient wird schließlich berechnet, indem der Korrelationskoeffizient nach Bravais und Pearson auf die transformierten Rangpaare angewendet wird. Der Wertebereich von $r_{SP}$ ist entsprechend gleich dem Wertebereich von $r$ und der berechnete Wert analog für die monotone Korrelation zu interpretieren. In der {\it NAG C Library} dient die Funktion nag\_ken\_spe\_corr\_coeff zur Berechnung dieses Korrelationskoeffizienten.

\subsection{Weiterführende Berechnungen}

Liegt eine Korrelation zwischen zwei Merkmalen vor, ist damit nicht bewiesen, dass auch ein \underline{kausaler} Zusammenhang vorliegt. Das Phänomen der sog. {\it Scheinkorrelation} besagt, dass eine Korrelation zwischen den Merkmale $X$ und $Y$ (z.B. Körpergröße und Wortschatz von Kindern) unter Umständen auf den Einfluss eines weiteren Merkmals $Z$ (z.B. Alter) zurückgeführt werden kann. Dies kann durch die Berechnung der partiellen Korrelation überprüft werden, wozu in der {\it NAG C Library} die Funktion nag\_partial\_corr dient. Diese und ggf. weitere Funktionen werden wir ebenfalls näher betrachten, sofern es der Rahmen dieser Arbeit erlaubt.