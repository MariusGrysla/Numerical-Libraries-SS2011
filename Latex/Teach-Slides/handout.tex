\documentclass[a4paper]{article}

\usepackage[utf8]{inputenc}
\usepackage{amsmath}
\usepackage[german]{babel}
\usepackage{amssymb}
\usepackage{amsxtra}
\usepackage[dvips]{epsfig,psfrag}
\usepackage{fullpage}
\usepackage[numbers]{natbib}

\bibliographystyle{plainnat}

\parindent0pt % verhindert einrücken nach Leerzeile

\title{
{\bf \scriptsize RHEINISCH-WESTF\"ALISCHE TECHNISCHE HOCHSCHULE AACHEN \\
LuFG Informatik 12 (Prof. Dr. rer. nat. Uwe Naumann)}\\
~\\
Seminar ``Numerische Bibliotheken''\\
{\bf \Large Korrelation und Regressionsanalyse} \\
{\large Handout} 
}

\author{Patrick Neidig, Marius Grysla}

\date{31. Mai 2011}

\begin{document}

\maketitle

Hallo!\cite{Cramer2007}\cite{Golub1989}

\section*{Beispiel}
Die verwendete Datensatz ist unter \cite{Fahrmeier2011} erhältlich.

\section*{Korrelation}

\section*{Multiple Regression}

{\bf Funktion:} nag\_regsn\_mult\_linear (g02dac)\cite{nag:intro}
\begin{itemize}
\item Führt eine multiple lineare Regression mit einem Regressanden und mehreren Regressoren durch.
\item Berechnet zusätzlich Standartfehler, Residuen und Statistiken zur Bewertung.
\end{itemize}

{\bf Allgemeiner Ansatz zur multiplen Regression:}
\begin{equation*}
  Y_i = \beta_0 + \beta_1 x_{i1} + \dots + \beta_p x_{ip} + \epsilon_i, \quad i = 1, \dots, n
\end{equation*}

{\bf Regressionsansatz zu Beispieldatensatz:}
\begin{equation*}
  log(NM) = \beta_0 + \beta_1 Wfl + \beta_2 Bj + \beta_3 gL + \epsilon
\end{equation*}
\begin{itemize}
\item Berechnung der Nettomiete nach Regression:
  \begin{equation*}
    NM = NM_B \alpha,\qquad
    NM_B = \beta_0 \times exp(\beta_1 Wfl) exp(\beta_2 Bj),\qquad
    \alpha = exp(\beta_3)
  \end{equation*}
\end{itemize}


{\bf Lösen des Ausgleichsproblems durch QR-Zerlegung mit Householder:}
\begin{itemize}
\item Householder Matrix (Multiplikation erzeugt Nullern in einer Spalte):
  \begin{equation*}
    P = I - 2vv^T / v^Tv
  \end{equation*}
\item Householder Vektor:
  \begin{equation*}
    v = x / \beta,\qquad v(1) = 1, \qquad \beta = x(1) + sign(x(1)) \|x\|_2
  \end{equation*}
\end{itemize}


\bibliography{../Report/report.bib}

\end{document}

%%% Local Variables: 
%%% mode: latex
%%% TeX-master: t
%%% End: 
