\documentclass[a4paper]{article}

\usepackage[utf8]{inputenc}
\usepackage{amsmath}
\usepackage[german]{babel}
\usepackage{amssymb}
\usepackage{amsxtra}
\usepackage[dvips]{epsfig,psfrag}
\usepackage{fullpage}
\usepackage[numbers]{natbib}

\bibliographystyle{plainnat}

\parindent0pt % verhindert einrücken nach Leerzeile

\title{
{\bf \scriptsize RHEINISCH-WESTF\"ALISCHE TECHNISCHE HOCHSCHULE AACHEN \\
LuFG Informatik 12 (Prof. Dr. rer. nat. Uwe Naumann)}\\
~\\
Seminar ``Numerische Bibliotheken''\\
{\bf \Large Korrelation und Regressionsanalyse} \\
{\large Handout} 
}

\author{Patrick Neidig, Marius Grysla}

\date{31. Mai 2011}

\begin{document}

\maketitle

Hallo!\cite{Cramer2007}\cite{Golub1989}

\section*{Beispiel}
\begin{itemize}
	\item Ausschnitt (2053 Fälle) aus Münchens Mietspiegel im Jahr 2003
	\item insbesondere zur Ermittlung der ortsüblichen Vergleichsmiete (Nettomiete in Abhängigkeit der Größe, Art, Ausstattung, Beschaffenheit und Lage der Wohnung) interessant
\end{itemize}

\par \textbf{Reduzierter Datensatz (8 Fälle ohne Variablen zur Ausstattung)}\\
	
	\begin{tabular}[ht]{|l|l|l|l|l|l|l|l|}
  	\hline
  	\textit{Nettomiete} & \textit{pro qm} & \textit{Wohnfläche} & \textit{Zimmer} & \textit{Baujahr} & \textit{Bezirk} & \textit{gute Wohnlage} & \textit{beste Wohnlage}\\
  	\hline \hline
  	214,14 & 5,95 & 36 & 2 & 1948 & 17 & 0 & 0\\ \hline
  	406,93 & 6,26 & 65 & 2 & 1966 & 10 & 0 & 0\\ \hline
		603,34 & 7,64 & 79 & 3 & 1986 & 13 & 1 & 0\\ \hline
		826,26 & 8,88 & 93 & 3 & 1918 & 5 & 1 &0\\ \hline
		1000,41 & 9,81 & 102 & 3 & 1918 & 8 & 0 & 0\\ \hline
		1227,12 & 10,14 & 121 & 5 & 1924 & 9 & 1 & 0\\ \hline
		1385,12 & 10,82 & 128 & 6 & 1991 & 9 & 1 & 0\\ \hline
		1661,55 & 11,08 & 150 & 5 & 1966 & 13 & 0 & 0\\
  	\hline
	\end{tabular} \\\\

Die gesamte verwendete Datensatz ist unter \cite{Fahrmeier2011} erhältlich.

\section*{Korrelation}

{\bf Funktion}: nag\_corr\_cov (g02bxc)\cite{nag:intro}
\begin{itemize}
	\item Frage: Welche Variablen haben Einfluss auf die Nettomiete?
	\item Lösung durch Algorithmus
		\begin{itemize}
			\item Eingabe: komplette Datenmatrix mit allen Variablen und Fällen
			\item sukzessive Berechnung der Erwartungswerte (analog zum arithmetischen Mittel, da ungewichtet) und Standardabweichungen für alle Variablen sowie Berechnung der Kovarianzmatrix
			\item abschließende Berechnung der Korrelationsmatrix (enthält die Korrelationskoeffizienten für alle Variablenpaare)
			\item Ausgabe: alle oben berechneten Werte
		\end{itemize}
	\end{itemize}

\section*{Multiple Regression}

{\bf Funktion:} nag\_regsn\_mult\_linear (g02dac)\cite{nag:intro}
\begin{itemize}
\item Führt eine multiple lineare Regression mit einem Regressanden und mehreren Regressoren durch.
\item Berechnet zusätzlich Standartfehler, Residuen und Statistiken zur Bewertung.
\end{itemize}

{\bf Allgemeiner Ansatz zur multiplen Regression:}
\begin{equation*}
  Y_i = \beta_0 + \beta_1 x_{i1} + \dots + \beta_p x_{ip} + \epsilon_i, \quad i = 1, \dots, n
\end{equation*}

{\bf Regressionsansatz zu Beispieldatensatz:}
\begin{equation*}
  log(NM) = \beta_0 + \beta_1 Wfl + \beta_2 Bj + \beta_3 gL + \epsilon
\end{equation*}
\begin{itemize}
\item Berechnung der Nettomiete nach Regression:
  \begin{equation*}
    NM = NM_B \alpha,\qquad
    NM_B = \beta_0 \times exp(\beta_1 Wfl) exp(\beta_2 Bj),\qquad
    \alpha = exp(\beta_3)
  \end{equation*}
\end{itemize}


{\bf Lösen des Ausgleichsproblems durch QR-Zerlegung mit Householder:}
\begin{itemize}
\item Householder Matrix (Multiplikation erzeugt Nullern in einer Spalte):
  \begin{equation*}
    P = I - 2vv^T / v^Tv
  \end{equation*}
\item Householder Vektor:
  \begin{equation*}
    v = x / \beta,\qquad v(1) = 1, \qquad \beta = x(1) + sign(x(1)) \|x\|_2
  \end{equation*}
\end{itemize}


\bibliography{../Report/report.bib}

\end{document}

%%% Local Variables: 
%%% mode: latex
%%% TeX-master: t
%%% End: 
